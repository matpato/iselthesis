%!TEX root = ../template.tex
%%%%%%%%%%%%%%%%%%%%%%%%%%%%%%%%%%%%%%%%%%%%%%%%%%%%%%%%%%%%%%%%%%%%
%% abstrac-pt.tex
%% ISEL thesis document file
%%
%% Abstract in Portuguese
%%%%%%%%%%%%%%%%%%%%%%%%%%%%%%%%%%%%%%%%%%%%%%%%%%%%%%%%%%%%%%%%%%%%
Independentemente da língua em que está escrita a dissertação, é necessário um resumo na língua do texto principal e um resumo noutra língua.  Assume-se que as duas línguas em questão serão sempre o Português e o Inglês.

O \emph{template} colocará automaticamente em primeiro lugar o resumo na língua do texto principal e depois o resumo na outra língua.  Por exemplo, se a dissertação está escrita em Português, primeiro aparecerá o resumo em Português, depois em Inglês, seguido do texto principal em Português. Se a dissertação está escrita em Inglês, primeiro aparecerá o resumo em Inglês, depois em Português, seguido do texto principal em Inglês.

O resumo não deve exceder uma página e deve responder às seguintes questões:
\begin{itemize}
% What's the problem?
	\item Qual é o problema?
% Why is it interesting?
	\item Porque é que ele é interessante?
% What's the solution?
	\item Qual é a solução?
% What follows from the solution?
	\item O que resulta (implicações) da solução?
\end{itemize}

\textbf{Máximo de palavras: 300}
% Palavras-chave do resumo em Português
\begin{keywords}
Palavras-chave (em Português) \ldots -- mínimo de 3 (três)
\end{keywords}
% to add an extra black line

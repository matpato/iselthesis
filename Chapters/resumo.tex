\abstractPT  % Do NOT modify this line

Independentemente da língua em que está escrita a dissertação, é necessário um resumo na língua do texto principal e um resumo noutra língua.  Assume-se que as duas línguas em questão serão sempre o Português e o Inglês. O \emph{template} colocará automaticamente em primeiro lugar o resumo na língua do texto principal e depois o resumo na outra língua.  
Resumo é a versão precisa, sintética e selectiva do texto do documento, destacando os elementos de maior importância. O resumo possibilita a maior divulgação da tese e sua indexação em bases de dados.

O resumo não deve conter citações bibliográficas, tabelas, quadros, esquemas. Dar preferência ao uso dos verbos na 3ª pessoa do singular. Tempo e verbo não devem dissociar-se dentro do resumo. Deve evitar o uso de abreviaturas e siglas - quando absolutamente necessário, citá-las entre parênteses e precedidas da explicação de seu significado, na primeira vez em que aparecem. 

E, deve-se evitar o uso de expressões como "O presente trabalho trata ...", "Nesta tese são discutidos....", "O documento conclui que....", "aparentemente é...." etc. 

% Keywords of abstract in Portuguese
\begin{keywords}
lista de palavras-chave em português
\end{keywords}
% to add an extra black line
